\chapter{Literature Review and Definitions}
\label{chap:lit_rev}

There are four main topics of research that support the work in this thesis: \ac{MDE}, \ac{PLE}, traceability, and requirement engineering. Individually, each of these topics have many branches of research, both academic and industrial. In order to narrow down the selection of work to support this thesis we aimed to find papers that explored at least two of the four topics. As bonus topics, we also considered papers if they involved a supporting tool or an exploration of adoption in industry. 

We also include some definitions in this chapter to establish nomenclature that will be used within the rest of this thesis. For \ac{PLE}, we consider a product line, product family, and feature model to be synonymous ways of capturing product information by connecting features together. Similarly, we consider model-based system engineering and model driven engineering as synonymous practices.

Traceability is scoped to requirement traceability and feature traceability, though not necessarily to each other. Much of the literature is focused on traceability itself with a smaller library of traceability between specific phases of development. As such we look for previous work that explores a variety of traceability methodologies and approaches. With these guiding topics we use the following strings to generate our search results:

%\section{Research Method}
%There were four main topics that were used to direct this literature review; \ac{MDE}, \ac{PLE}, traceability, and requirement engineering. This thesis had explored different ways to support traceability activities between current \ac{PLE} techniques and requirement engineering. The \ac{MDE} aspect of this work comes in the form of both tool support and fundamental descriptions of the methodology. As such when looking for related or previous iterations, we focused on \ac{MDE} activities that support traceability or/and requirement engineering. The \ac{PLE} and \ac{MDE} domains already have a lot of overlap. As such we focused more on tooling and specification for \ac{PLE}. Thus, the search strings used for this literature review are as follows:
\begin{itemize}
	\item (``model driven engineering" OR ``model based engineering") AND (``product line engineering" OR ``feature modelling") AND ``tools"
	\item (``model driven engineering" OR ``model based engineering") AND ``traceability" AND ``tools"
	\item ``traceability" AND (``product line engineering" OR ``feature modelling") AND ``tools"
	\item ``traceability" AND (``requirement modelling" OR ``requirement diagrams") AND ``tools"
\end{itemize}

Results using these search strings were chosen from the initial search results from Google Scholar and the following conferences:
\begin{itemize}
	\item International Conference on Software Engineering (ICSE). 
	\item International Conference on Model Driven Engineering Languages and Systems (MODELS). 
	\item Software Product Line Conference (SPLC).
	\item Variability Modelling of Software-Intensive Systems (VaMoS).
\end{itemize}

\section{Inclusion Criteria}

In order to deem a resulting publication to be relevant to this body of work the following criteria needed to be met:
\begin{itemize}
	\item The publication should use \ac{MDE} techniques AND address \ac{PLE} OR requirement engineering. The publication can focus on development in any applied domain, though of particular interest are automotive and medical device development. 
	\begin{itemize}
		\item Of particular interest are applications of feature modelling in industry. 
		\item Of particular interest are application of \ac{MDE} techniques for requirement engineering and modelling.
	\end{itemize}
	\item The publication should explore traceability. This can be done either for requirements OR \ac{PLE} though ideally the publication should explore traceability explicitly between \ac{PLE} and requirements. This also includes methodologies for automated traceability.
	\item The publication should have tool support for traceability, \ac{PLE}, or/and requirement modelling. 
\end{itemize}

We parse the results and separate them into the three domains; \ac{PLE}, traceability, and requirement engineering. Though we setup \ac{MDE} as one of the topics for this thesis, it is more of a generalized one as we explore \ac{PLE}, traceability, and requirement engineering through the lens of \ac{MDE}. \ac{MDE} is also used to generate tool support for the task of feature modelling, traceability, and requirement engineering.

Another reason for separating the results in this manner is to identify synonymous definitions and specifications across domains. Looking at the various topics through the lens of \ac{MDE} allows us to find patterns of similarity across the various publication. 


%Ties to the \ac{MDE} is one of the boundaries for the results. There is a combination of formalism and informal approaches to analyzing and solving the problem of traceability between product families and requirements. We also include some fundamental works to support definitions and understandings within the various domains. This is to help support one of our hypothesizes that various definitions across the domains are synonymous. 

As part of this work we explore two domains; the automotive domain and medical device domain. We want to show that the proposed methodology and implementation can be effective in those two domains as they have safety-critical requirements. As such we also include some literature for development in the automotive industry and medical devices, specifically pacemakers for the latter. For the supporting literature however we were more broad and looked across multiple industries to see what the state of the practice is for \ac{PLE}, traceability, and requirement engineering.

%Traceability is a highly studied topic in both academia and industry. For the scope of this paper, we are looking at related work that handles traceability through product lines, approaches that support some ability to iterate on traceability, and previously defined templates for traceability matrices. We include product lines in our related work because they are commonly used to support iterative and incremental development. We also include some previous work around requirement engineering.




%One pattern we found within the literature is that traceability is most commonly considered a retrospective task to be completed after development. However we firmly believe that traceability should not be reserved for retrospective documentation, it should be a part of development itself. Thus we created the traceability matrix in a requirement modelling environment. The traceability matrix was loosely inspired by the matrix of the CDC~\cite{tmat_cdc}.
%
%For one, \tool\ is a Domain-Specific Language (DSL) developed using Epsilon~\cite{kolovos2010epsilon}, Xtext~\cite{eysholdt2010xtext}, Sirius~\cite{viyovic2014sirius}, and the Eclipse Modelling Framework (EMF). Another difference is that the traceability matrix representation created in \tool\ has more detail relative to the former as we connected requirements with design elements and testing elements. More information about the mechanisms and matrices in \tool\ are shown in section~\ref{section:tool_impl}. The biggest difference is that the traceability matrices generated are part of a much larger system that will integrate feature modelling with our requirements canvases to generate feature-requirement traceability. 


%Cleland-Huang and colleagues have an extensive body of work focused on developing methods and tools for automated and automating traceability, \textit{e.g.}, \cite{Cleland-HuangBCSR07}. Some of this work targets safety-critical domains and explores traceability between assurance cases and system design artifacts \cite{AgrawalC23}. She has also tackled traceability in a product line for her Dronology project~\cite{DBLP:conf/splc/Cleland-HuangAI20}.




\section{Product Line Engineering}

\ac{PLE} has become a popular approach to handling diversity and complexity in a companies product portfolio since its original conception by Kang et al.~\cite{kang1990feature, kang2002feature}. We scoped the related work to some fundamentals of \ac{PLE} and some concrete implementations (tool support). As one of the objectives for this thesis is to add little extensions to the existing theories of product families we paid special attention to formal specifications and existing tools.

Many tools exist for feature modelling. Two examples of tools come in FeatureIDE and GEARS. FeatureIDE~\cite{kastner2009featureide, thum2014featureide}, developed originally by K\"{a}stner and Th{\"u}m et al, is a feature modelling plugin tool for Eclipse that lets a user define their product software as a feature model, write code within the model, and seamlessly integrate the features together to form an executable product. It has more recently become the research community standard as a feature modelling tool. Another example for feature modelling and PLE is GEARS~\cite{GEARS} by Big Lever, which emphasizes its 3-tiered software product line methodology. The 3-tiered approach outlines three different ways to use GEARS: handling variation and automation within a single product line, an approach to identifying reusable assets across product lines, and product evolution over time. 

As a formal approach to \ac{PLE}, feature modelling can also be defined through algebraic methods as shown in the work of Peter H\"{o}fner et al.~\cite{hofner2006feature,hofner2011algebra} which uses set theory as the basis for proving how their algebra works. This algebra can be extremely useful for tool development that implements feature modelling as it allows for the computability of feature models, thus allowing for easier implementations of feature modelling and product line engineering tools.

Another direction that feature modelling has taken can be found in the work of Czarnecki et al.~\cite{czarnecki2004staged}. While still involving formalisms for the development of feature models, there is more focus on the graphical modelling of feature models as opposed to mathematical specification and computability. They expand on the original syntax defined in FODA and show by using context-free grammar (in this case metamodelling) how to add cardinality to feature models and their relevance to feature modelling paradigms. This specification's main benefit is handling feature duplication in contrast to the work of H\"{o}fner et al.

We know that there is a desire for better \ac{PLE} practices for industry. In a survey paper by Horcas et al.~\cite{horcas2019software} they conclude a lack of tool support for the various approaches to \ac{PLE} as roadblock to more ubiquitous adoption in industry. This is further supported by the workshop conducted by Becker et al.~\cite{becker2024not} where they identified 7 themes to the challenges of \ac{PLE}across various industries: Modularity, Product Line Scoping, Variability Management, General Process, Verification and Validation, Evolution and Maintenance, and Key Performance Indicators. There is some ongoing work to address some of these issues. Bilic at al.~\cite{bilic2019integrated} present an \ac{MDE} tool chain and methodology to manage variability in system design. Co-evolution is another problem that Jongeling et al~\cite{jongeling2020co} present a solution for in Simulink models in a product line. For the automotive industry specifically, we can see some initial benefits based on an implementation by Wozniak et al.~\cite{10.1145/2791060.2791071} and their usage of GEARS. Their act of capturing their product line in a consistent model based environment was helpful for managing complexity. The ability to automate configuration calibrations were a great benefit for their purposes as well. As these were just their initial efforts, they also called for more research and work to be done in this domain with respect to industry practices. These publications highlight that a gap exists between academic and industrial practices. 

\section{Requirement Engineering and Modelling}

We consider requirement engineering as the act of eliciting, refining, and specifying requirements for a given item, be it a product, feature, or component. However, we found a gap in the literature with respect to \ac{MDE} techniques and requirement engineering, specifically around requirement modelling. This will be a significant contribution of this thesis as we explore this gap and identify how \ac{MDE} techniques can enhance existing requirement engineering approaches.

We also explore informal and formal approaches to requirement engineering. They both have merits in different situations, with informal methods being more approachable, while the formal methods inspire more confidence due to their natural rigour. We also found various definitions for functional and non-functional requirements that were helpful for informing our own design decisions about how we approach requirements in this thesis.

%Requirement engineering is a critical portion of this research. Part of the contribution of this thesis is the advancement of requirement modelling. As such we look for related works that have come before with efforts to combine \ac{MDE} with requirement engineering.

The idea of using feature models to encapsulate requirements was partially inspired by the requirement engineering book by Bertrand Meyer~\cite{meyer2022handbook}. His style emphasizes a lot of stakeholder analysis for determining their use cases and scoping system requirements around user interactions. It emphasizes a more informal approach to requirement engineering, lowering the bar for entry and making it more approachable, while recommending the occasional ``formal picnic" as he refers to it as needed. The purpose of the ``formal picnic" is to supplement requirement decisions and specifications with more rigorous methodologies as needed, rather than mandating formal methods be adhered to throughout the requirement engineering process.

We also have Axel Van Lamsweerde's style of requirements~\cite{lamsweerde2009requirements}. There is a lot of overlap that exists between Meyer requirements and Van Lamsweerde requirements, however, there exist some key differences in the ease of implementation of the requirements processes. Van Lamsweerde's requirement style dives deeper into the formalism and MDE approaches to requirement engineering, in contrast to Meyer's requirement style. Van Lamsweerde used formal goal diagrams as a way to capture knowledge about the stakeholders of a product and to justify the emerging requirements from the goals. Another difference is in the definition of a goal. Van Lamsweerde defines a goal as a ``prescriptive statement of intent that the system should satisfy through the cooperation of its agents." In contrast, a goal is defined by Meyer as the ``needs of the target organization, which the system will address". To generalize, Van Lamsweerde's style requirements are a more in-depth version of Meyer's requirements due to the more formal approach to requirement engineering, at the cost of the approachability found in Meyer's style of requirements.

%We focused on MDE requirement engineering techniques, feature modelling strategies and approaches, and other work related to traceability in product lines. To remain within the scope of requirement engineering, we focused on requirement engineering resources that focus more on general requirement engineering than specific requirement notations. 

Due to the gap we discovered in requirement modelling, we focused many of our efforts on other requirement engineering resources that did not include \ac{MDE} as a main component of their approach. Suzanne and James Robertson~\cite{robertson2012mastering, robertson2000volere}, heavily emphasize stakeholder identification and involvement early on in the project. There is more emphasis on business use cases and functional vs. non-functional requirements in contrast to Meyer's requirements which focuses more on the use cases for requirements concerning stakeholders. Meyer's requirements style is also structured with four `books' and each book represents a different portion of the requirement engineering process, each with a suggested structure for organizing their respective section. By contrast, Robertson's requirement engineering approach focuses more on how requirements are written, and their specific categorizations. The requirement engineering process of Sawyer and Sommerville~\cite{sommerville1997requirements, sommerville1997viewpoints} discusses the different viewpoints within system requirements. Their work stresses that not all stakeholders are people, and could be anything from an organization to another system to be integrated/interacted with. There are many more that could be talked about, however, we can generalize a pattern:
\begin{itemize}
	\item Identify who/what you are developing for.
	\item Identify what/how the stakeholder interacts with a target system.
	\item Specify the behaviour around that interaction.
\end{itemize}

While requirement diagrams from SysML~\cite{sysml2019omg} are a helpful approach for requirements engineering and modelling, there are other modelling schemes that have been explored in this topic. Use case diagrams from UML~\cite{fowler2018uml}, have been explored by many people for their value in requirements engineering. Siau and Lee ultimately concluded that use case diagrams helped communicate requirements compared to other modelling~\cite{siau2004use}, but have limited usability beyond communication. They have also been used by von der Ma{\ss}en and Horst Lichter~\cite{von2002modeling} for software product line development, having to customize the metamodel to make it work. Wegmann and Genilloud~\cite{wegmann2000role} formalized use cases and use case diagrmas for functional requirements for systems, though they faced difficulty in scaling the model to handle more complex systems in detail and demonstrate limitations for use case diagrams. FORML, a Feature Oriented Requirement Modelling Language by Joanne Atlee and colleagues~\cite{Beidu2019, 6345799}, is another approach to feature scoped requirement based on product families. They use superimposed state machines to express behavioral requirements of the features that compose a product line.

%Another related requirement engineering approach includes the work from Suzanne and James Robertson~\cite{robertson2012mastering, robertson2000volere}, which has many helpful definitions for functional and non-functional requirements and their relation to business cases, which are similar to Meyer's use cases. Pete Sawyer and Ian Sommerville~\cite{sommerville1997requirements, sommerville1997viewpoints} have another requirement engineering approach which discusses how system viewpoints affect the scope and definitions of requirements. Finally, Axel Van Lamsweerde's requirement engineering style~\cite{lamsweerde2009requirements} uses a similar approach to requirement engineering as Meyer's but has a much greater emphasis on the use of formal methods for requirement elicitation and specification. While useful for proving requirement specification, this goes beyond the goal of this paper as we focus primarily on the implementation of \tool\ and the methodology that it supports.


\section{Traceability}

Traceability is a commonly studied topic in academia and industry. We found that traceability was commonly considered a retrospective task of product development due to the difficulty and cost in maintaining it throughout development. We found this is due to a common philosophy of separating traceability as a task from the development process instead of including as a natural outcome of development. This has led to a variety of post-release analysis techniques to build traceability artifacts.

As this thesis aims to produce traceability as a natural part of development instead of a retrospective task, we explored various ways of both automated and manual approaches to traceability. We kept the scope of traceability artifacts broad to help explore traceability throughout the phases of development beyond the feature and requirement phases.


%For the scope of this thesis, we explore traceability as it is used in \ac{PLE} and \ac{MDE}. We further scope related work based on how traceability is used with requirements. 

The concept behind this paper shares some similarities to the work done by Marques et al.~\cite{6945504}. While we both aim to automate traceability matrix generation and maintenance by using requirements diagrams, there are distinct differences in our implementations and end goals. We focused our efforts into distinct \ac{MDE} environment (requirement modelling, feature modelling, use case diagrams, and goal diagrams). Meanwhile, the approach from Marques et al. aims to generate traceability between requirement, design, and testing.

Dronology, the work of Cleland-Huang and colleagues, is a project that has produced comprehensive work around and including traceability. As part of their project they explore traceability through the lens of safety assurance~\cite{Cleland-HuangBCSR07, AgrawalC23, DBLP:conf/splc/Cleland-HuangAI20, mirakhorli2011tracing}.

%They cover automation and tooling for drone development and use traceability artifacts as part of their safety assurance activities~\cite{Cleland-HuangBCSR07, AgrawalC23, DBLP:conf/splc/Cleland-HuangAI20, mirakhorli2011tracing}.

The work of Heisig et al~\cite{heisig2019generic} proposes a generic traceability metamodel for end-to-end traceability in software product lines. Their proposed method is supposed to facilitate and enable comprehensive traceability throughout the development process from requirements to implementation across multiple MDE tools. As such they developed a traceability metamodel to handle a variety of artifacts they anticipate will require traceability links. In contrast, our work does not focus on developing a traceability metamodel. Instead, we focus on mapping requirements to features and generating traceability from the mappings. Another example of traceability in product lines is in the work of Tsuchiya et al.~\cite{tsuchiya2013recovering}. Their work focuses on recovering lost traceability links due to incremental development. They propose a framework and implement a tool that shows potential in recovering lost traceability links due to natural software maintenance over time. This is relevant as it is another approach to traceability maintenance through incremental software changes. Where they focus on recovering lost links, we attempt to update the links parallel to changes.

%Heisig \textit{\textit{et al}}~\cite{heisig2019generic} focused on using a generic traceability metamodel for software product lines. They concluded that using a metamodel facilitated the traceability of various artifacts throughout development. 

Similarly, the work of Kelleher~\cite{kelleher2005reusable} also required the definition of a traceability metamodel, though this was developed as a means to handle complexity in software systems as opposed to product lines. Another approach can be found in the work of Maletic et al~\cite{maletic2005xml}, whose primary focus was flexibility and interoperability of traceability artifacts, allowing for the direct modelling of artifacts through model transformations using XML. Asuncion et al.~\cite{asuncion2010software} used machine learning to generate traceability prospectively and retrospectively; during development and after development.


\section{Discussion}

Having gone through a variety of publications exploring \ac{MDE}, \ac{PLE}, requirement engineering, and traceability we notice a few patterns. While there are some \ac{MDE} techniques that are used to help with requirement engineering, both formal and informal, there is a lack of literature about modelling requirements themselves. Specifically, the use of requirement diagrams from SysML is an under-explored theme in comparison to more popular modelling strategies like \ac{UCD}s, and to a lesser extent goal diagrams. 

There are many different strategies for handling traceability, from machine learning to manual processes. We also found some previous work for generating traceability in a software product line. We found that we share a similar philosophy to Asuncion et al. that traceability should be a prospective task rather than a retrospective one. While we share that philosophy, we approach it very differently as we do not use machine learning as part of our traceability generation. We consider traceability should be a natural outcome of good engineering, emerging naturally as various phases of development are completed.

We generally agree with Meyer's approach to requirement engineering in supporting a more informal and flexible approach to requirements, with some detours into more rigorous methods to supplement and support requirement specification as needed. This is the main reason why we follow a similar development path that he presents in his handbook.

Finally, \ac{PLE} and industry require more support for state of the practice. We've identified across multiple industries a desire for better \ac{PLE} in industry and many of the difficulties they face with implementing it. While we do not aim to solve all of them in this thesis, we do aim to lower the bar for entry in implementing \ac{PLE} techniques for industry, and creating tools to help support these efforts.

%\section{Automotive}
%
%\ac{PLE} is a relatively new methodology for the automotive industry. 
%
%\section{Medical Devices}

