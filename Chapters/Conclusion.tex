\chapter{Conclusion}


What comes first, the requirement or the feature? In this thesis we have attempted to outline an answer to this question. Neither, it is the customer use cases and the user goals that come first. Those allow us to identify features and elicit requirements in parallel. We have shown that following Meyer's requirements engineering approach we can leverage both informal and formal \ac{MDE} techniques to perform a domain and problem space analysis. We can identify who our customers/users/stakeholders are, what we expect them to do with our system, and why they would want to use our system to satisfy their goals. We have shown to we can map the identified use cases to features of our system. We have shown how we can derive high-level requirement from our goal diagram and how we can refine and decompose those requirements. Thus we have shown satisfaction of RQ\ref{RQ:4}.

Thanks to this process contribution, we are also able to identify features and elicit requirements in parallel. The next step we need to do is decide which features own which requirements. This is first done at a high-level as we map which features satisfy a user goal or goals. Then we can refine those goals into requirements that are owned by the feature, all scoped by the feature to ensure relevance. We have shown an added bonus to this approach using the feature decomposition in the feature model. This implies a natural decomposition in the requirements as well as they are scoped by the features that own them. Overall, there is a much clearer path to specifying features in a \ac{FDD} environment with the supporting methodology and tool. This satisfies RQ\ref{RQ:1}.

One of the main highlights of our methodology is the feature-requirement encapsulation. By defining this hierarchy, we have shown how it facilitates traceability between features and requirements. We can show what features own the requirements, dependency between features, and by extension, dependencies external features and requirements. We have demonstrated how this hierarchy enables increased granularity of traceability through our implementation of CyclicL. We were able to define a formal metamodel to capture this hierarchy and implement a tool to show satisfiability of this proposal. Through CyclicL, we exposed the benefit of this hierarchical relationship between features and requirements in the semi-automated maintenance and generation of traceability matrices between features and requirements, satisfying RQ\ref{RQ:2}.

Finally, CyclicL has shown the potential of a \ac{PLE} tool that is self contained with requirements as part of the tool. We have shown that we can support incremental and iterative development of both feature models and requirement models in CyclicL. By leveraging \ac{EMF} and Sirius, we were able to show the possibilities enabled by a tool that will maintain traceability through both requirement and architectural changes without dependencies on external support. While there are still some limitations in the current tool capabilities, we have demonstrated how future development can continue to address these short-comings. Therefore, we believe that we have satisfied RQ\ref{RQ:3} and the potential of a tool that supports iterative and incremental development of traceability in parallel to feature and requirement development.


