\chapter{Conclusion}


What comes first, the requirement or the feature? In this thesis we have attempted to outline an answer to this question. incidentally, and perhaps unsurprisingly, the answer is both and it depends. Both realities are possibilities for companies in industry as they startup new product development from scratch or try to reuse previous work for a new model. So we support both of them. 

The ultimate answer however, is neither. The stakeholder use cases and goals come first as we conduct a domain analysis and problem space exploration. We have shown that following Meyer's requirements engineering approach we can leverage both informal and formal \ac{MDE} techniques to perform a domain and problem space analysis. We found that goal diagrams were an easy to use, informal approach to capturing knowledge of what stakeholders really want to do which helped to inform why they would want to use a product we develop. The \ac{UCD} was important for establishing how the would use our product. Identifying what interfaces the stakeholders would need from the product to accomplish their goals. Goal diagrams and \ac{UCD}s formed the core of our informal modelling techniques for domain analysis and problem space exploration.

Feature modelling and our requirement canvas leveraged more formal approaches to modelling and problem space exploration as with \tool\ we could enforce syntactic and semantic rules when creating the models. Equating features and use cases also helped to establish requirements for our features as we refined the use cases into user stories. With our proposed feature-requirement hierarchy, those requirements are immediately scoped and owned by the features making an easy transition from problem space exploration to requirement specification. We have demonstrated both formal and informal approaches to domain analysis, problem space exploration, and requirement development while using \ac{MDE} techniques. We have explored several processes for accomplishing these tasks in multiple environments. Finally, we have demonstrated how this methodology can be used to identify gaps in existing documentation and how to improve requirements specification and identification. Thus, we conclude that we have satisfied~\ref{RQ4}

% who our customers/users/stakeholders are, what we expect them to do with our system, and why they would want to use our system to satisfy their goals. We have shown to we can map the identified use cases to features of our system. We have shown how we can derive high-level requirement from our goal diagram and how we can refine and decompose those requirements. Thus we have shown satisfaction of~\ref{RQ4}.



%Thanks to this process contribution, we are also able to identify features and elicit requirements in parallel or in either order. The next step we need to do is decide which features own which requirements. This is first done at a high-level as we map goals to use cases, and then equate use cases to features. Then we can refine those use cases into requirements through user stories that are owned and scoped by the feature to ensure relevance. We have shown an added bonus to this approach in the implicit ordering that emerges in the functional requirements. This implied order has the potential to help with project planning in \ac{FDD} development environment and helps ensure that the features are well defined. This satisfies~\ref{RQ1}.

\ac{FDD} is a well established development strategy. A major contribution of this thesis is the proposed feature-requirement hierarchy that establishes features own their requirements. Another contribution is how we have equated use cases from \ac{UCD}s to features in \ac{PLE}. Our methodology refines the use cases into user stories to create our requirements for each use case. Due to having equated use cases and features, those requirements are thus also applicable to the feature. When combined with the proposed feature-requirement hierarchy we end up with well defined requirements for each feature. An added bonus to this we discovered was a natural ordering that formed in the requirements due to the nature of using user stories as a refinement process. These all contribute to having well defined and scoped features, and potentially aiding in project planning due to the implied ordering of requirements from the user stories. These are all possible ways to enhance \ac{FDD} development environments, satisfying~\ref{RQ1}



%using the feature decomposition in the feature model. This implies a natural decomposition in the requirements as well as they are scoped by the features that own them. Overall, there is a much clearer path to specifying features in a \ac{FDD} environment with the supporting methodology and tool. 

%map which features satisfy a user goal or goals. Then we can refine those goals into requirements that are owned by the feature, all scoped by the feature to ensure relevance. We have shown an added bonus to this approach using the feature decomposition in the feature model. This implies a natural decomposition in the requirements as well as they are scoped by the features that own them. Overall, there is a much clearer path to specifying features in a \ac{FDD} environment with the supporting methodology and tool. This satisfies~\ref{RQ1}.

One of the main contributions of our methodology is the feature-requirement encapsulation. By defining this hierarchy, we have shown how it facilitates traceability between features and requirements. We also removed ambiguity in the relationship between features and requirements. We have demonstrated how this hierarchy enables increased granularity of traceability through our implementation of CyclicL. We were able to define a formal metamodel to capture this hierarchy and implement a tool to show satisfiability of this proposal. Through CyclicL, we exposed the benefit of this hierarchical relationship between features and requirements in the semi-automated maintenance and generation of traceability matrices between features and requirements. This was completed as a by product of following the methodology and naturally using \tool\ to create models and specify our requirements. We thus got the traceability matrices without any manual intervention or extra steps in the engineering process. We conclude that this reduces much of the time and effort that would traditionally have been spent manually creating or maintaining engineering data traceability, satisfying~\ref{RQ2}.

%We can show what features own the requirements. 

Finally, CyclicL has shown the potential benefits of a \ac{PLE} tool that is self contained with requirements as part of the tool in the context of iterative and incremental development. We have shown that we can support incremental development of both feature models and requirement models in CyclicL. By leveraging \ac{EMF} and Sirius, we were able to show the possibilities enabled by a tool that will maintain traceability through both requirement changes and product changes without dependencies on external support. While there are still some limitations in the current tool capabilities to support iterative changes, we have demonstrated how future development can continue to address these short-comings. Due to the currently limitations in the tool, we conclude that we have partially satisfied~\ref{RQ3} with \tool\ as a tool that supports iterative and incremental development of traceability in parallel to feature and requirement development. We give it a partial designation as there are still limitations on how \tool\ can support iterative changes under our current definition of a change between development phases.

Overall our methodology has shown demonstrable improvements on an existing requirement document, provided avenues for potential improvements in requirement engineering and feature modelling, and shown potential for integrating with existing development practice. \tool and its supporting methodology are ripe for further exploration and continued development to support safety-critical development in further domains and potentially as a more generalized approach to requirement engineering across industries.

