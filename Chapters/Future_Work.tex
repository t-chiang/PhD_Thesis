\chapter{Future Work}

While the results of our proposed methodology and tooling are promising, there are some definite short-comings at present. While CyclicL is self-contained with regards to features and requirements, there is no traceability between the \ac{UCD} and goal diagram to the tool. This presents another challenge of traceability and justification. It requires the engineer to be aware of the links between the domain and problem space analysis and the feature and requirement models in the tool. This implicit knowledge will eventually need to be documented somewhere to support both future development changes and assurance for safety-critical development. As a result, one potential path is creating a bridge between the domain analysis modelling and the current tool to extend the traceability all the way up to the concept phase of development. While we expect \ac{UCD} and goal diagrams to remain mostly static once they are created, the direct traceability could be helpful documentation for onboarding new employees to help them get familiar with why certain decisions were made, support change impact analysis justifications, and aid the development of assurance case by having complete end-to-end traceability between the problem space and the solution space.

The implementation of the tool, CyclicL, is still incomplete as we have only demonstrated the capability of the minimum viable product. There are many functions we will need to add to have a more complete tool. There are currently no syntax constraints in the tool to ensure that all user created models are valid according to the metamodel. Thus there is potential for users to create semantically meaningless models even if the syntax is shown as correct. We are currently relying on conventions and good engineering to create semantically meaningful models but this is very difficult to maintain at scale within a company or industry. Another lane of development would be in allowing users to create these models textually as well as graphically. Very often adoptability of a tool, especially an \ac{MDE} tool is difficult as many engineers are unfamiliar with \ac{MDE} techniques and approaches. Creating a textual environment for creating these models may help with usability and approachability of CyclicL for use in industry. Most importantly, we would want to implement a method of consistency checking within the tool. This would remove a lot of the burden from the engineer to ensure that their features are consistent and their requirement are consistent between features. This is a unique challenge as the requirements are specified using the Gherkin behavioral approach and thus may require us to re-evaluate what type of specification language we implement within the requirement canvases of CyclicL.



