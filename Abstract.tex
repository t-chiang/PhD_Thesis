\chapter*{Abstract}

%An abstract of not more than 300 words must be included and will indicate the major emphasis of the thesis, new discoveries, and its contribution to knowledge.
Product Line Engineering (PLE) has become a common engineering practice for managing system complexity across multiple industries. The practice is used for identifying reusable pieces of code, composing software components together, and modelling product variation. There exists a gap however with generating system traceability when using PLE as a technique. Specifically, it is a colossal problem to get traceability from a feature model, the modelling environment for PLE, through requirements to design elements. Further, traceability itself is both a costly and tedious task that is often completed at the end of development and even more difficult to maintain throughout a products life-cycle. With the methodology proposed in this thesis, along with its supporting tool \tool, we aim to address the gap that exists between PLE and requirement engineering to push traceability out of a retrospective task to an active portion of development.