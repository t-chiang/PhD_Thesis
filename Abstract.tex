\chapter*{Abstract}

%An abstract of not more than 300 words must be included and will indicate the major emphasis of the thesis, new discoveries, and its contribution to knowledge.
Product Line Engineering (PLE) has become a common approach in engineering in the context managing product complexity and reuse across multiple industries. The practice is used for identifying reusable pieces of code, composing software components, and modelling product variation. There exists a gap however with generating system traceability when using PLE. Specifically, it is a challenging to generate traceability from a feature model through requirements to design elements. Further, traceability itself is both a costly and tedious task that is often completed at the end of development in order to meet certification standards or for assurance purposes. At the same time, supporting traceability in a PLE context is difficult as products go through their life-cycle. This thesis proposes a methodology to improve the identification of features through \ac{MDE} techniques, define a hierarchical relationship between features and requirements, and provide a tool supported approach to generating and maintaining traceability in a PLE context throughout a product life-cycle. The contributions of this thesis will attempt to address the gap that exists between \ac{PLE} and requirement engineering to migrate traceability out of a retrospective task towards a more active portion of development.
